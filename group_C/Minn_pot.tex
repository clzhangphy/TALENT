\documentclass[10pt,a4paper]{article}
\usepackage[utf8]{inputenc}
%\usepackage{lmodern}
\usepackage[T1]{fontenc}
\usepackage{amsmath}
\usepackage{amsfonts}
\usepackage[pdftex,unicode]{hyperref}
\hypersetup{pdftitle=Minnesota potential in laboratory frame}
\hypersetup{pdfauthor=Anton Repko}
\setlength{\oddsidemargin}{-25pt}
\setlength{\topmargin}{-30pt}
\setlength{\headheight}{0pt}
\setlength{\headsep}{0pt}
\addtolength{\textheight}{150pt}
\addtolength{\textwidth}{160pt}

\author{A.~Repko}
\title{Minnesota potential in laboratory frame}

\begin{document}

\section{Minnesota potential}
Antisymmetrized Minnesota potential is
\begin{equation}
\label{Minnesota_ini}
\widetilde{V}(\vec{r}_1,\vec{r}_2) = \big[ \hat{V}_R(\vec{r}_1,\vec{r}_2)
+\tfrac{1}{2}(1+\hat{P}_\sigma)\hat{V}_t(\vec{r}_1,\vec{r}_2)
+\tfrac{1}{2}(1-\hat{P}_\sigma)\hat{V}_s(\vec{r}_1,\vec{r}_2) \big]
\times\tfrac{1}{2}(1+\hat{P}_r)(1-\hat{P}_\sigma\hat{P}_r)
\end{equation}
where $\hat{P}_\sigma$ is spin-exchange and $\hat{P}_r$ space-exchange operator. The parts of the interaction are of gaussian type:
\begin{subequations}
\begin{align}
\hat{V}_R(\vec{r}_1,\vec{r}_2) &= +V_{0,R}\mathrm{e}^{-\kappa_R(\vec{r}_1-\vec{r}_2)^2},&
V_{0,R} &=200.00\ \mathrm{MeV},& \kappa_R &=1.487\ \mathrm{fm^{-2}} \\
\hat{V}_t(\vec{r}_1,\vec{r}_2) &= -V_{0,t}\mathrm{e}^{-\kappa_t(\vec{r}_1-\vec{r}_2)^2},&
V_{0,t} &=178.00\ \mathrm{MeV},& \kappa_t &=0.639\ \mathrm{fm^{-2}} \\
\hat{V}_s(\vec{r}_1,\vec{r}_2) &= -V_{0,s}\mathrm{e}^{-\kappa_s(\vec{r}_1-\vec{r}_2)^2},&
V_{0,s} &=91.85\ \mathrm{MeV},& \kappa_s &=0.465\ \mathrm{fm^{-2}}
\end{align}
\end{subequations}
Due to antisymmetry, the space-exchange operator can be converted to spin-exchange and vice versa:
\begin{equation}
(1+\hat{P}_r)(1-\hat{P}_\sigma\hat{P}_r) = 1 + \hat{P}_r - \hat{P}_\sigma\hat{P}_r - \hat{P}_\sigma = (1-\hat{P}_\sigma)(1-\hat{P}_\sigma\hat{P}_r)
\end{equation}
and because of $(1+\hat{P}_\sigma)(1-\hat{P}_\sigma)=0$, the term $\hat{V}_t$ gives zero contribution and can be omitted. The final form is then
\begin{equation}
\label{Minnesota}
\widetilde{V}(\vec{r}_1,\vec{r}_2) = \tfrac{1}{2}\big[ \hat{V}_R(\vec{r}_1,\vec{r}_2)
+\hat{V}_s(\vec{r}_1,\vec{r}_2) \big]
(1-\hat{P}_\sigma)(1-\hat{P}_\sigma\hat{P}_r)
\end{equation}

\section{Multipolar decomposition of the Gaussian interaction}
References in the following text beginnging with ``V.'' point to the formulas from the book D.~A.~Varshalovich, A.~N.~Moskalev, V.~K.~Khersonskii: Quantum Theory of Angular Momentum, World Scientific Singapore 1988. Following symmetry properties of Clebsch-Gordan coefficients will be used without further reference:
\begin{equation}
C_{l,m,\lambda,\mu}^{J,M} = (-1)^{l+\lambda-J} C_{l,-m,\lambda,-\mu}^{J,-M}
= C_{\lambda,-\mu,l,-m}^{J,-M} =
(-1)^{l-m}\sqrt{\frac{2J+1}{2\lambda+1}}\,C_{l,m,J,-M}^{\lambda,-\mu} =
(-1)^{\lambda+\mu}\sqrt{\frac{2J+1}{2l+1}}\,C_{J,-M,\lambda,\mu}^{l,-m}
\tag{V.~8.4.10,11}
\end{equation}

Since Talmi-Moshinsky transformation to center-of-mass frame is analytically difficult, I will factorize a two-body Gaussian interaction in laboratory frame as
\begin{equation}
\mathrm{e}^{-\mu(\vec{r}_1-\vec{r}_2)^2} = \mathrm{e}^{-\mu(r_1^2+r_2^2)}
\mathrm{e}^{2\mu\vec{r}_1\cdot\vec{r}_2}
\end{equation}
The term with scalar product can be decomposed using a well known formula
\begin{equation}
\mathrm{e}^{\mathrm{i}\vec{k}\cdot\vec{r}} = 4\pi\sum_{LM} \mathrm{i}^L
j_L(kr) Y_{LM}^*(\hat{k}) Y_{LM}(\hat{r})
\tag{V.~5.17.14}
\end{equation}
where imaginary unit is eliminated by utilizing modified spherical Bessel function $i_L(z)$ instead of spherical Bessel function $j_L(z)$:
\begin{equation}
(-\mathrm{i})^L j_L(iz) = i_L(z) = i_L^*(z) = \sum_{k=0}^\infty
\frac{z^{L+2k}}{2^k k! (2L+2k+1)!!}
\end{equation}
Desired formula is therefore
\begin{equation}
\label{gaus_dec}
\mathrm{e}^{-\mu(\vec{r}_1-\vec{r}_2)^2} = \mathrm{e}^{-\mu(r_1^2+r_2^2)}
4\pi\sum_{LM}i_L(2\mu r_1 r_2)Y_{LM}^*(\hat{r}_1)Y_{LM}(\hat{r}_2) =
\mathrm{e}^{-\mu(r_1-r_2)^2}4\pi\sum_{LM}
\frac{i_L(2\mu r_1 r_2)}{\exp(2\mu r_1 r_2)} Y_{LM}^*(\hat{r}_1)Y_{LM}(\hat{r}_2)
\end{equation}
Last expression takes into account exponentially divergent nature of $i_L(z)$ for large $z$. In fact, the version of $i_L(z)$ given by the GSL library is exactly  the one rescaled by an exponential.

Sometimes, only the $L=0$ part is needed. It can be expressed analytically by well known functions ($Y_{00} = 1/\sqrt{4\pi}$):
\begin{equation}
i_0(z) = \frac{\mathrm{e}^z - \mathrm{e}^{-z}}{2z} \quad\rightarrow\quad
\Big(\mathrm{e}^{-\mu(\vec{r}_1-\vec{r}_2)^2}\Big)_{L=0} =
\frac{\mathrm{e}^{-\mu(r_1-r_2)^2}-\mathrm{e}^{-\mu(r_1+r_2)^2}}{4\mu r_1 r_2}
4\pi Y_{00}^*(\hat{r}_1) Y_{00}(\hat{r}_2)
\end{equation}

\section{Matrix elements of the particle-hole part (for HF)}
Interaction is rewritten in second quantization as
\begin{equation}
\hat{V} = \tfrac{1}{4}\bar{v}_{acbd}\hat{a}_a^+\hat{a}_c^+\hat{a}_d\hat{a}_b
\end{equation}
where the labeling was chosen to give a convenient expression for the mean-field HF hamiltonian:
\begin{equation}
\label{HF_hamil}
h_{ab} = t_{ab} + \sum_{cd}\bar{v}_{acbd}\rho_{dc} \qquad\textrm{where }
\rho_{dc} = \langle\mathrm{HF}|\hat{a}_c^+\hat{a}_d|\mathrm{HF}\rangle
\end{equation}
Matrix element $\bar{v}_{acbd}$ is then calculated according to (\ref{Minnesota}) symbolically as
\begin{subequations}\label{v_dec}
\begin{align}
\label{v_dec1}
\bar{v}_{acbd} &= \big[\psi_a^\dagger(r_1\sigma_1)\psi_b(r_1\sigma_1)\big]V_{12}
\big[\psi_c^\dagger(r_2\sigma_2)\psi_d(r_2\sigma_2)\big]
- \big[\psi_a^\dagger(r_1\sigma_1)\psi_d(r_1\sigma_1)\big]V_{12}
\big[\psi_c^\dagger(r_2\sigma_2)\psi_b(r_2\sigma_2)\big] \\
\label{v_dec2}
&\quad - \big[\psi_a^\dagger(r_1\sigma_1)\psi_b(r_1\sigma_2)\big]V_{12}
\big[\psi_c^\dagger(r_2\sigma_2)\psi_d(r_2\sigma_1)\big]
+ \big[\psi_a^\dagger(r_1\sigma_1)\psi_d(r_1\sigma_2)\big]V_{12}
\big[\psi_c^\dagger(r_2\sigma_2)\psi_b(r_2\sigma_1)\big]
\end{align}
\end{subequations}
where the single particle wavefunction of spherical HO with quantum numbers $a=(n_a,j_a,l_a,m_a)$ is ($\chi$ is spinor)
\begin{equation}
\label{spin-orbit}
\psi_a(\vec{r}) = R_{n_a l_a}(r) \sum_\sigma C_{l_a,m_a-\sigma,\frac{1}{2},\sigma}^{j_a,m_a} Y_{l_a,m_a-\sigma}(\hat{r}) \chi_\sigma
\end{equation}
and $V_{12}$ is a shortcut for $\tfrac{1}{2}V_{R,s}(\vec{r}_1,\vec{r}_2)$. Integration over $\vec{r}_1,\,\vec{r}_2$ and summation over $\sigma_1,\,\sigma_2$ is assumed in (\ref{v_dec}) without explicit notion.

The terms in (\ref{v_dec}) will not be evaluated straight away, but only after considering the summation over $m$ of $cd$ in (\ref{HF_hamil}), and will be highlighted by $\textbf{[+]}$ and $\textbf{[--]}$. The point is that density matrix $\rho_{cd}$ mixes only the states with the same $(j,l,m)$, assuming unbroken spherical symmetry. Moreover, the density matrix is numerically the same for all $m$. The density matrix can be therefore broken into independent submatrices $\rho_{cd}^{(j,l)}$, where $j_c=j_d=j$, $l_c=l_d=l$ and $m_c=m_d$ is arbitrary (i.e.~there are $2j+1$ identical submatrices with different $m$).

Analytical evaluation of $\sum_{m}\bar{v}_{acbd}\rho_{dc}$ for individual terms in (\ref{v_dec}) will go as follows:
\begin{enumerate}
\item Angular functions in $\big[\ldots\big]$ will be coupled into $Y_{LM}(r_{1/2})$, which will be then eliminated by angular integration (by orthogonality of $Y_{LM}$) with corresponding term from $V_{12}$. The second bracket will be first treated by relation $Y_{L,-M} = (-1)^M Y_{LM}^*$. The form of (\ref{v_dec1}) allows the summation over $\sigma_{1,2}$ to be performed in this step as well.
\item Summation over $m_c=m_d=m$ with $\rho_{cd}^{(j,l)}$ will be applied in the second bracket. Besides forcing $j_c=j_d=j$ and $l_c=l_d=l$, this will lead to various simplification and equalities, often in a multi-step way. An important feature is the equality of $LM$ for the first and second bracket $\big[\ldots\big]$ due to (\ref{gaus_dec}).
\item Finally, the equalities $j_a=j_b$, $l_a=l_b$ and $m_a=m_b$ (as Kronecker deltas) will be obtained together with final expressions. The final expression will still contain radial integrals over $r_1,\,r_2$ and summation over $L$. Finally, the density matrix $\rho_{cd}^{(j,l)}$ can be formally factorized out from the result to obtain effective matrix elements $\bar{v}_{acbd}$.
\end{enumerate}

When the coupling of spatial and spin part is not separated, spinorbitals can be directly multiplied to get
\begin{align}
\big[\psi_a^\dagger(r_1\sigma_1)\psi_b(r_1\sigma_1)\big] &=
R_a(r_1)R_b(r_1)\sum_{LM}(-1)^{j_a+m_a+j_b+L+\frac{1}{2}}
\sqrt{\frac{(2j_a+1)(2j_b+1)(2l_a+1)(2l_b+1)}{4\pi(2L+1)}} \nonumber\\
\label{psi_ab} &\qquad\qquad\qquad\qquad
\times\begin{Bmatrix} j_a & j_b & L \\ l_b & l_a & \frac{1}{2} \end{Bmatrix}
C_{l_a 0 l_b 0}^{L 0} C_{j_a,-m_a,j_b,m_b}^{L,M} Y_{LM}(\hat{r}_1)
\tag{V.~7.2.40}
\end{align}
Summation of the second bracket with density matrix leads to
\begin{align}
\sum_m\big[\psi_c^\dagger(r_2\sigma_2)\psi_d(r_2\sigma_2)\big]\rho_{dc}^{(j,l)} &=
\rho_{dc}^{(j,l)} R_c(r_2) R_d(r_2) \sum_L
\Big[\sum_m(-1)^{j+m}C_{j,-m,j,m}^{L,0}\Big] \nonumber\\
&\qquad\times\frac{(2j+1)(2l+1)}{\sqrt{4\pi(2L+1)}}
\begin{Bmatrix} j & j & L \\ l & l & \frac{1}{2} \end{Bmatrix}
(-1)^{j+L+\frac{1}{2}} C_{l 0 l 0}^{L 0} Y_{L0}(\hat{r}_2)
\end{align}
This expression can be simplified by the following steps (orthogonality of CG coefficients etc.):
\begin{equation}
(-1)^{j+m} = \sqrt{2j+1}\,C_{j,-m,j,m}^{0,0},\quad
\sum_m(-1)^{j+m}C_{j,-m,j,m}^{L,0} = \sqrt{2j+1}\,\delta_{L0}
\tag{V.~8.5.1, 8.7.2 or 4}
\end{equation}
\begin{equation}
C_{l0l0}^{00} = \frac{(-1)^l}{\sqrt{2l+1}},\quad
\begin{Bmatrix} j & j & 0 \\ l & l & \frac{1}{2} \end{Bmatrix} =
\frac{(-1)^{j+l+\frac{1}{2}}}{\sqrt{(2l+1)(2j+1)}}
\tag{V.~9.5.1}
\end{equation}
to get
\begin{equation}
\sum_m\big[\psi_c^\dagger(r_2\sigma_2)\psi_d(r_2\sigma_2)\big]\rho_{dc}^{(j,l)} =
(2j+1)\rho_{dc}^{(j,l)} R_c(r_2) R_d(r_2) \frac{1}{\sqrt{4\pi}}Y_{00}^*(\hat{r}_2)
\end{equation}
From $L=0$, I get $j_a=j_b$, $l_a=l_b$, and $m_a=m_b$ in (\ref{psi_ab}) due to Clebsch-Gordan coefficients (or $6j$ symbol), using also $C_{j,-m,j,m}^{0,0} = (-1)^{j+m}/\sqrt{2j+1}$:
\begin{align}
\sum_m &\big[\psi_a^\dagger(r_1\sigma_1)\psi_b(r_1\sigma_1)\big]V_{12}
\big[\psi_c^\dagger(r_2\sigma_2)\psi_d(r_2\sigma_2)\big]\rho_{dc}^{(j,l)} =\nonumber\\
\label{HFterm1}
&= (2j+1)\rho_{dc}^{(j,l)} \int_0^\infty r_1^2\mathrm{d}r_1
\int_0^\infty r_2^2\mathrm{d}r_2 R_a(r_1) R_b(r_1)
\frac{\mathrm{e}^{-\mu(r_1-r_2)^2}-\mathrm{e}^{-\mu(r_1+r_2)^2}}{4\mu r_1 r_2}
R_c(r_2) R_d(r_2) \qquad \textbf{[+]}
\end{align}

Exchange term in (\ref{v_dec1}) is more complicated:
\begin{align}
\sum_m &\big[\psi_a^\dagger(r_1\sigma_1)\psi_d(r_1\sigma_1)\big]V_{12}
\big[\psi_c^\dagger(r_2\sigma_2)\psi_b(r_2\sigma_2)\big]\rho_{dc}^{(j,l)} =\nonumber\\
&= \rho_{dc}^{(j,l)} \int_0^\infty r_1^2\mathrm{d}r_1
\int_0^\infty r_2^2\mathrm{d}r_2 R_a(r_1) R_d(r_1)
\mathrm{e}^{-\mu(r_1-r_2)^2} R_c(r_2) R_b(r_2)
\sum_{LM}\frac{i_L(2\mu r_1 r_2)}{\exp(2\mu r_1 r_2)}
\frac{(-1)^{j_a-j_b+m_a-m_b}}{2L+1} \nonumber\\
&\quad\times \sqrt{(2j_a+1)(2j_b+1)(2l_a+1)(2l_b+1)}\,(2j+1)(2l+1)
\begin{Bmatrix} j_a & j & L \\ l & l_a & \frac{1}{2} \end{Bmatrix}
\begin{Bmatrix} j & j_b & L \\ l_b & l & \frac{1}{2} \end{Bmatrix}
C_{l_a 0 l 0}^{L0} C_{l 0 l_b 0}^{L0} \nonumber\\
&\quad\times \sum_m C_{j_a,-m_a,j,m}^{L,M} C_{j,-m,j_b,m_b}^{L,-M}
\end{align}
Sum over $m$ and $M$ in two last CG coefficients is
\begin{equation}
\sum_{mM} C_{j_a,-m_a,j,m}^{L,M} C_{j,-m,j_b,m_b}^{L,-M} = \frac{2L+1}{\sqrt{(2j_a+1)(2j_b+1)}}\sum_{mM} C_{L,-M,j,m}^{j_a,m_a} C_{L,-M,j,m}^{j_b,m_b}
= \frac{2L+1}{2j_a+1} \delta_{j_a j_b} \delta_{m_a m_b}
\end{equation}
Then, the coefficients $C_{l_a 0 l 0}^{L0} C_{l 0 l_b 0}^{L0}$ imply that both $l_a,\,l_b$ are either even or odd. So they are equal (due to $j=l\pm\tfrac{1}{2}$). Final result is
\begin{align}
\sum_m &\big[\psi_a^\dagger(r_1\sigma_1)\psi_d(r_1\sigma_1)\big]V_{12}
\big[\psi_c^\dagger(r_2\sigma_2)\psi_b(r_2\sigma_2)\big]\rho_{dc}^{(j,l)} =\nonumber\\
&= (2j+1)\rho_{dc}^{(j,l)} \int_0^\infty r_1^2\mathrm{d}r_1
\int_0^\infty r_2^2\mathrm{d}r_2 R_a(r_1) R_d(r_1)
\mathrm{e}^{-\mu(r_1-r_2)^2} R_c(r_2) R_b(r_2) \nonumber\\
&\qquad\times (2l_a+1)(2l+1)\sum_L \frac{i_L(2\mu r_1 r_2)}{\exp(2\mu r_1 r_2)}
\begin{Bmatrix} j_a & j & L \\ l & l_a & \frac{1}{2} \end{Bmatrix}^2
\Big(C_{l_a 0 l 0}^{L0}\Big)^2 \qquad \textbf{[--]}
\end{align}

Terms with spin-exchanged coordinates (\ref{v_dec2}) need to be decomposed to spatial and spin part (\ref{spin-orbit}). Spatial part will be coupled first:
\begin{equation}
Y_{l_a,m_a-\sigma_1}^*(\hat{r}_1) Y_{l_b,m_b-\sigma_2}(\hat{r}_1) =
\sum_{LM} (-1)^{m_a-\sigma_1}
\sqrt{\frac{(2l_a+1)(2l_b+1)}{4\pi(2L+1)}}\,C_{l_a 0 l_b 0}^{L 0}
C_{l_a,-m_a+\sigma_1,l_b,m_b-\sigma_2}^{L,M}
Y_{LM}(\hat{r}_1)
\tag{V.~5.6.9}
\end{equation}
and spin part $\sigma_{1,2}$ will be coupled and eliminated later. Second bracket of the first term in (\ref{v_dec2}) after summation with $\rho_{dc}$ gives:
\begin{align}
\sum_m \big[\psi_c^\dagger(r_2\sigma_2)\psi_d(r_2\sigma_1)\big]\rho_{dc}^{(j,l)} &=
\rho_{dc}^{(j,l)} R_c(r_2)R_d(r_2)\sum_L \frac{2l+1}{\sqrt{4\pi(2L+1)}}
C_{l0l0}^{L0} Y_{L,\sigma_2-\sigma_1}(\hat{r}_2) \nonumber\\
\label{HFterm3a}
&\quad\times \sum_m (-1)^{m-\sigma_2} C_{l,m-\sigma_2,\frac{1}{2},\sigma_2}^{j,m}
C_{l,m-\sigma_1,\frac{1}{2},\sigma_1}^{j,m}
C_{l,-m+\sigma_2,l,m-\sigma_1}^{L,\sigma_2-\sigma_1}
\end{align}
The sum of CG coefficients in the last line can be converted into a $6j$ symbol:
\begin{align}
\sum_m (-1)^{m-\sigma_2} &C_{l,-m+\sigma_2,l,m-\sigma_1}^{L,\sigma_2-\sigma_1}
C_{l,m-\sigma_2,\frac{1}{2},\sigma_2}^{j,m}
C_{l,m-\sigma_1,\frac{1}{2},\sigma_1}^{j,m} = \nonumber\\
&=(-1)^{j+\frac{1}{2}} \sqrt{\frac{2L+1}{2}}\,(2j+1)
C_{L,\sigma_2-\sigma_1,\frac{1}{2},\sigma_1}^{\frac{1}{2},\sigma_2}
\begin{Bmatrix} l & l & L \\ \frac{1}{2} & \frac{1}{2} & j \end{Bmatrix}
\tag{V.~8.7.16} \\
&= (-1)^{l}\frac{2j+1}{2\sqrt{2l+1}}\delta_{\sigma_1\sigma_2}
\end{align}
The last equality is based on the following: Triangular inequality in the obtained CG coefficient allows an $L$ equal to 0 or 1. So, according to $C_{l0l0}^{L0}$ in (\ref{HFterm3a}), $L$ is even, and thus $L=0$ and $\sigma_1 = \sigma_2$. Then, $l_a = l_b$ follows from $C_{l_a 0 l_b 0}^{00}$, $m_a = m_b$ follows from $C_{l_a,-m_a+\sigma,l_a,m_b-\sigma}^{0,0}$, and $j_a = j_b$ follows from
\begin{equation}
\sum_{\sigma} C_{l_a,m_a-\sigma,\frac{1}{2},\sigma}^{j_a,m_a}
C_{l_a,m_a-\sigma,\frac{1}{2},\sigma}^{j_b,m_a} = \delta_{j_a j_b}
\tag{V.~8.7.4}
\end{equation}
Final result for the third term of (\ref{v_dec}) is then (it is half of the first term (\ref{HFterm1}))
\begin{align}
\sum_m &\big[\psi_a^\dagger(r_1\sigma_1)\psi_b(r_1\sigma_2)\big]V_{12}
\big[\psi_c^\dagger(r_2\sigma_2)\psi_d(r_2\sigma_1)\big]\rho_{dc}^{(j,l)} =\nonumber\\
&= \frac{1}{2} (2j+1)\rho_{dc}^{(j,l)} \int_0^\infty r_1^2\mathrm{d}r_1
\int_0^\infty r_2^2\mathrm{d}r_2 R_a(r_1) R_b(r_1)
\frac{\mathrm{e}^{-\mu(r_1-r_2)^2}-\mathrm{e}^{-\mu(r_1+r_2)^2}}{4\mu r_1 r_2}
R_c(r_2) R_d(r_2) \qquad \textbf{[--]}
\end{align}

Last term is more tricky, since simplifications are possible only when the term is written in its entirety:
\begin{align}
\sum_m &\big[\psi_a^\dagger(r_1\sigma_1)\psi_d(r_1\sigma_2)\big]V_{12}
\big[\psi_c^\dagger(r_2\sigma_2)\psi_b(r_2\sigma_1)\big]\rho_{dc}^{(j,l)} =\nonumber\\
&= \rho_{dc}^{(j,l)}\int_0^\infty r_1^2\mathrm{d}r_1
\int_0^\infty r_2^2\mathrm{d}r_2 R_a(r_1) R_d(r_1)
\mathrm{e}^{-\mu(r_1-r_2)^2} R_c(r_2) R_b(r_2) \\
&\quad\times\sum_L \frac{i_L(2\mu r_1 r_2)}{\exp(2\mu r_1 r_2)}
\frac{\sqrt{(2l_a+1)(2l_b+1)}\,(2l+1)}{2L+1}
C_{l_a 0 l 0}^{L0} C_{l 0 l_b 0}^{L0} (-1)^{m_a-m_b} \nonumber\\
&\quad\times\sum_{mM\sigma_1\sigma_2}
C_{l_a,m_a-\sigma_1,\frac{1}{2},\sigma_1}^{j_a,m_a}
C_{l,m-\sigma_2,\frac{1}{2},\sigma_2}^{j,m}
C_{l_a,-m_a+\sigma_1,l,m-\sigma_2}^{L,M}
C_{l,m-\sigma_2,\frac{1}{2},\sigma_2}^{j,m}
C_{l_b,m_b-\sigma_1,\frac{1}{2},\sigma_1}^{j_b,m_b}
C_{l,-m+\sigma_2,l_b,m_b-\sigma_1}^{L,-M} \nonumber
\end{align}
Last summation contains four degrees of freedom (in fact, only three degrees of freedom give nontrivial contribution) which have to be treated carefully. First I will sum over $m$ and $\sigma_2$, keeping constant $m-\sigma_2$ (thus eliminating one degree of freedom), then over $m-\sigma_2$ and $M$, and finally over $\sigma_1$:
\begin{subequations}
\begin{align}
\sum_{m,\sigma_2}^{(1)} C_{l,m-\sigma_2,\frac{1}{2},\sigma_2}^{j,m}
C_{l,m-\sigma_2,\frac{1}{2},\sigma_2}^{j,m} &=
\frac{2j+1}{2l+1}\sum_{m,\sigma_2}^{(1)}
C_{j,-m,\frac{1}{2},\sigma_2}^{l,-m+\sigma_2}
C_{j,-m,\frac{1}{2},\sigma_2}^{l,-m+\sigma_2} = \frac{2j+1}{2l+1} \\
\sum_{m-\sigma_2,M} C_{l_a,-m_a+\sigma_1,l,m-\sigma_2}^{L,M}
C_{l,-m+\sigma_2,l_b,m_b-\sigma_1}^{L,-M} &= \frac{2L+1}{\sqrt{(2l_a+1)(2l_b+1)}}
\sum_{m-\sigma_2,M} C_{L,-M,l,m-\sigma_2}^{l_a,m_a-\sigma_1}
C_{L,-M,l,m-\sigma_2}^{l_b,m_b-\sigma_1} \nonumber\\
&= \frac{2L+1}{2l_a+1}\delta_{l_a l_b} \delta_{m_a m_b} \\
\sum_{\sigma_1} C_{l_a,m_a-\sigma_1,\frac{1}{2},\sigma_1}^{j_a,m_a}
C_{l_a,m_a-\sigma_1,\frac{1}{2},\sigma_1}^{j_b,m_a} &= \delta_{j_a j_b}
\end{align}
\end{subequations}
The fourth term of (\ref{v_dec}) is then
\begin{align}
\sum_m &\big[\psi_a^\dagger(r_1\sigma_1)\psi_d(r_1\sigma_2)\big]V_{12}
\big[\psi_c^\dagger(r_2\sigma_2)\psi_b(r_2\sigma_1)\big]\rho_{dc}^{(j,l)} =\nonumber\\
&= (2j+1)\rho_{dc}^{(j,l)} \int_0^\infty r_1^2\mathrm{d}r_1
\int_0^\infty r_2^2\mathrm{d}r_2 R_a(r_1) R_d(r_1)
\mathrm{e}^{-\mu(r_1-r_2)^2} R_c(r_2) R_b(r_2)
\sum_L \frac{i_L(2\mu r_1 r_2)}{\exp(2\mu r_1 r_2)}
\Big(C_{l_a 0 l 0}^{L0}\Big)^2 \quad \textbf{[+]}
\end{align}

\section{Matrix elements of the particle-particle part (for HFB)}
Pairing part of the HFB mean-field is
\begin{equation}
\Delta_{ab} = \frac{1}{2}\sum_{cd} \bar{v}_{abcd}\kappa_{cd} \qquad\textrm{where }
\kappa_{cd}=\langle\mathrm{HFB}|\hat{a}_d\hat{a}_c|\mathrm{HFB}\rangle
\end{equation}
Independence of $\kappa_{cd}$ on $m$ has to be treated more carefully, it is not given a priori, also because $m_c = -m_d$. It turns out that independence on $m$ can be implemented by using time-reversed states (I use convention $\hat{T} = \mathrm{i}\hat{\sigma}_y\hat{K}$, where $\hat{K}$ is complex conjugation):
\begin{equation}
\psi_{\bar{a}} = \hat{T}\psi_a = \begin{pmatrix} 0 & 1 \\ -1 & 0 \end{pmatrix}
\psi_a^* = R_{n_a,l_a}\sum_\sigma C_{l_a,m_a-\sigma,\frac{1}{2},\sigma}^{j_a,m_a}
(-1)^{m_a-\sigma} Y_{l_a,-m_a+\sigma} (-1)^{\sigma+\frac{1}{2}} \chi_{-\sigma}
= (-1)^{l_a+j_a+m_a}\psi_{-a}
\end{equation}
where $-a=(n_a,j_a,l_a,-m_a)$. So an $m$-independent pairing tensor can be defined as
\begin{equation}
\kappa_{c\bar{d}}^{(j,l)} = (-1)^{l+j+m}\kappa_{c,-d} \qquad(m=m_c=m_d)
\end{equation}
Matrix element of the interaction can be then calculated in four parts
\begin{subequations}\label{vpair_dec}
\begin{align}
\label{vpair_dec1}
\bar{v}_{a\bar{b}c\bar{d}} &= \big[\psi_a^\dagger(r_1\sigma_1)\psi_c(r_1\sigma_1)\big]V_{12}
\big[\psi_{\bar{b}}^\dagger(r_2\sigma_2)\psi_{\bar{d}}(r_2\sigma_2)\big]
- \big[\psi_a^\dagger(r_1\sigma_1)\psi_{\bar{d}}(r_1\sigma_1)\big]V_{12}
\big[\psi_{\bar{b}}^\dagger(r_2\sigma_2)\psi_c(r_2\sigma_2)\big] \\
\label{vpair_dec2}
&\quad -\big[\psi_a^\dagger(r_1\sigma_1)\psi_c(r_1\sigma_2)\big]V_{12}
\big[\psi_{\bar{b}}^\dagger(r_2\sigma_2)\psi_{\bar{d}}(r_2\sigma_1)\big]
+ \big[\psi_a^\dagger(r_1\sigma_1)\psi_{\bar{d}}(r_1\sigma_2)\big]V_{12}
\big[\psi_{\bar{b}}^\dagger(r_2\sigma_2)\psi_c(r_2\sigma_1)\big]
\end{align}
\end{subequations}

The first term is
\begin{align}
\sum_m &\big[\psi_a^\dagger(r_1\sigma_1)\psi_c(r_1\sigma_1)\big]V_{12}
\big[\psi_{\bar{b}}^\dagger(r_2\sigma_2)\psi_{\bar{d}}(r_2\sigma_2)\big]
\kappa_{c\bar{d}}^{(j,l)} = \nonumber\\
&\quad =\kappa_{c\bar{d}}^{(j,l)} \int_0^\infty r_1^2\mathrm{d}r_1
\int_0^\infty r_2^2\mathrm{d}r_2 R_a(r_1) R_c(r_1) \mathrm{e}^{-\mu(r_1-r_2)^2}
R_b(r_2) R_d(r_2) \sum_{LM}\frac{i_L(2\mu r_1 r_2)}{\exp(2\mu r_1 r_2)}
\frac{(2j+1)(2l+1)}{2L+1} \nonumber\\
&\qquad\times\sqrt{(2j_a+1)(2j_b+1)(2l_a+1)(2l_b+1)}\,
\begin{Bmatrix} j_a & j & L \\ l & l_a & \frac{1}{2} \end{Bmatrix}
\begin{Bmatrix} j_b & j & L \\ l & l_b & \frac{1}{2} \end{Bmatrix}
C_{l_a 0 l 0}^{L0} C_{l_b 0 l 0}^{L0} \nonumber\\
&\qquad\times\sum_m (-1)^{j_a-j_b+m_a+m}(-1)^{l_b+l+j_b+j+m_b+m}
C_{j_a,-m_a,j,m}^{L,M} C_{j_b,m_b,j,-m}^{L,-M}
\end{align}
Sum over $m$ and $M$ is evaluated as
\begin{equation}
\sum_{mM} C_{j_a,-m_a,j,m}^{L,M} C_{j_b,m_b,j,-m}^{L,-M} =
\frac{(-1)^{j_b+j-L} (2L+1)}{\sqrt{(2j_a+1)(2j_b+1)}} \sum_{mM}
C_{L,-M,j,m}^{j_a,m_a} C_{L,-M,j,m}^{j_b,m_b} =
(-1)^{j_a+j+L} \frac{2L+1}{2j_a+1} \delta_{j_a j_b} \delta_{m_a m_b}
\end{equation}
Equality $l_a=l_b$ then follows from parity in $C_{l_a 0 l 0}^{L0} C_{l_b 0 l 0}^{L0}$. Final result is
\begin{align}
\sum_m &\big[\psi_a^\dagger(r_1\sigma_1)\psi_c(r_1\sigma_1)\big]V_{12}
\big[\psi_{\bar{b}}^\dagger(r_2\sigma_2)\psi_{\bar{d}}(r_2\sigma_2)\big]
\kappa_{c\bar{d}}^{(j,l)} = \nonumber\\
&\quad =(2j+1)\kappa_{c\bar{d}}^{(j,l)} \int_0^\infty r_1^2\mathrm{d}r_1
\int_0^\infty r_2^2\mathrm{d}r_2 R_a(r_1) R_c(r_1) \mathrm{e}^{-\mu(r_1-r_2)^2}
R_b(r_2) R_d(r_2) \nonumber\\
&\qquad\times(2l_a+1)(2l+1) \sum_L\frac{i_L(2\mu r_1 r_2)}{\exp(2\mu r_1 r_2)}
\begin{Bmatrix} j_a & j & L \\ l & l_a & \frac{1}{2} \end{Bmatrix}^2
\Big(C_{l_a 0 l 0}^{L0}\Big)^2 \qquad \textbf{[+]}
\end{align}

The second term is
\begin{align}
\sum_m &\big[\psi_a^\dagger(r_1\sigma_1)\psi_{\bar{d}}(r_1\sigma_1)\big]V_{12}
\big[\psi_{\bar{b}}^\dagger(r_2\sigma_2)\psi_c(r_2\sigma_2)\big]
\kappa_{c\bar{d}}^{(j,l)} = \nonumber\\
&\quad =\kappa_{c\bar{d}}^{(j,l)} \int_0^\infty r_1^2\mathrm{d}r_1
\int_0^\infty r_2^2\mathrm{d}r_2 R_a(r_1) R_d(r_1) \mathrm{e}^{-\mu(r_1-r_2)^2}
R_b(r_2) R_c(r_2) \sum_{LM}\frac{i_L(2\mu r_1 r_2)}{\exp(2\mu r_1 r_2)}
\frac{(2j+1)(2l+1)}{2L+1} \nonumber\\
&\qquad\times\sqrt{(2j_a+1)(2j_b+1)(2l_a+1)(2l_b+1)}\,
\begin{Bmatrix} j_a & j & L \\ l & l_a & \frac{1}{2} \end{Bmatrix}
\begin{Bmatrix} j_b & j & L \\ l & l_b & \frac{1}{2} \end{Bmatrix}
C_{l_a 0 l 0}^{L0} C_{l_b 0 l 0}^{L0} \nonumber\\
&\qquad\times\sum_m (-1)^{j_a-j_b+m_a-m}(-1)^{l+l_b+j+j_b+m+m_b}
C_{j_a,-m_a,j,-m}^{L,M} C_{j_b,m_b,j,m}^{L,-M} \nonumber\\
&\quad= -(2j+1)\kappa_{c\bar{d}}^{(j,l)} \int_0^\infty r_1^2\mathrm{d}r_1
\int_0^\infty r_2^2\mathrm{d}r_2 R_a(r_1) R_d(r_1) \mathrm{e}^{-\mu(r_1-r_2)^2}
R_b(r_2) R_c(r_2) \nonumber\\
&\qquad\times(2l_a+1)(2l+1) \sum_L\frac{i_L(2\mu r_1 r_2)}{\exp(2\mu r_1 r_2)}
\begin{Bmatrix} j_a & j & L \\ l & l_a & \frac{1}{2} \end{Bmatrix}^2
\Big(C_{l_a 0 l 0}^{L0}\Big)^2 \qquad \textbf{[--]}
\end{align}

The third term is
\begin{align}
\sum_m &\big[\psi_a^\dagger(r_1\sigma_1)\psi_c(r_1\sigma_2)\big]V_{12}
\big[\psi_{\bar{b}}^\dagger(r_2\sigma_2)\psi_{\bar{d}}(r_2\sigma_1)\big]
\kappa_{c\bar{d}}^{(j,l)} = \nonumber\\
&\quad =\kappa_{c\bar{d}}^{(j,l)} \int_0^\infty r_1^2\mathrm{d}r_1
\int_0^\infty r_2^2\mathrm{d}r_2 R_a(r_1) R_c(r_1) \mathrm{e}^{-\mu(r_1-r_2)^2}
R_b(r_2) R_d(r_2) \sum_{LM}\frac{i_L(2\mu r_1 r_2)}{\exp(2\mu r_1 r_2)}
\frac{2l+1}{2L+1} \nonumber\\
&\qquad\times\sqrt{(2l_a+1)(2l_b+1)}\,
C_{l_a 0 l 0}^{L0} C_{l_b 0 l 0}^{L0} \sum_{m\sigma_1\sigma_2}
(-1)^{m_a+m+l_b+l+j_b+j+m_b+m} \nonumber\\
&\qquad\times C_{l_a,m_a-\sigma_1,\frac{1}{2},\sigma_1}^{j_a,m_a}
C_{l,m-\sigma_2,\frac{1}{2},\sigma_2}^{j,m}
C_{l_a,-m_a+\sigma_1,l,m-\sigma_2}^{L,M}
C_{l_b,-m_b-\sigma_2,\frac{1}{2},\sigma_2}^{j_b,-m_b}
C_{l,-m-\sigma_1,\frac{1}{2},\sigma_1}^{j,-m}
C_{l_b,m_b+\sigma_2,l,-m-\sigma_1}^{L,-M}
\end{align}
The sum can be converted into a $9j$ symbol:
\begin{align}
\sum_{mM\sigma_1\sigma_2}\!\!\! &
C_{l_a,-m_a+\sigma_1,l,m-\sigma_2}^{L,M}
C_{l,-m-\sigma_1,\frac{1}{2},\sigma_1}^{j,-m}
C_{l_a,m_a-\sigma_1,\frac{1}{2},\sigma_1}^{j_a,m_a}
C_{l_b,m_b+\sigma_2,l,-m-\sigma_1}^{L,-M}
C_{l,m-\sigma_2,\frac{1}{2},\sigma_2}^{j,m}
C_{l_b,-m_b-\sigma_2,\frac{1}{2},\sigma_2}^{j_b,-m_b} = \nonumber\\
&\quad=\sum_{mM\sigma_1\sigma_2}
(-1)^{-\sigma_2-\sigma_1+l+\frac{1}{2}-j+l_b+l-L+\sigma_1+\sigma_2+l_b+\frac{1}{2}-j_b}
\frac{(2L+1)(2j+1)}{2\sqrt{(2l_a+1)(2l_b+1)}} \nonumber\\
&\qquad\qquad\times C_{L,-M,l,m-\sigma_2}^{l_a,m_a-\sigma_1}
C_{l,m+\sigma_1,j,-m}^{\frac{1}{2},\sigma_1}
C_{l_a,m_a-\sigma_1,\frac{1}{2},\sigma_1}^{j_a,m_a}
C_{L,-M,l,m+\sigma_1}^{l_b,m_b+\sigma_2}
C_{l,m-\sigma_2,j,-m}^{\frac{1}{2},-\sigma_2}
C_{l_b,m_b+\sigma_2,\frac{1}{2},-\sigma_2}^{j_b,m_b} \nonumber\\
&\quad=(-1)^{L+j-j_b} (2L+1)(2j+1)
\begin{Bmatrix} L & l & l_a \\ l & j & \frac{1}{2} \\
l_b & \frac{1}{2} & j_a \end{Bmatrix}\delta_{j_a j_b}\delta_{m_a m_b}
\tag{V.~10.1.8}
\end{align}
Equality $l_a=l_b$ then follows from $C_{l_a 0 l 0}^{L0} C_{l_b 0 l 0}^{L0}$. The final result is
\begin{align}
\sum_m &\big[\psi_a^\dagger(r_1\sigma_1)\psi_c(r_1\sigma_2)\big]V_{12}
\big[\psi_{\bar{b}}^\dagger(r_2\sigma_2)\psi_{\bar{d}}(r_2\sigma_1)\big]
\kappa_{c\bar{d}}^{(j,l)} = \nonumber\\
&\quad =-(2j+1)\kappa_{c\bar{d}}^{(j,l)} \int_0^\infty r_1^2\mathrm{d}r_1
\int_0^\infty r_2^2\mathrm{d}r_2 R_a(r_1) R_c(r_1) \mathrm{e}^{-\mu(r_1-r_2)^2}
R_b(r_2) R_d(r_2) \nonumber\\
&\qquad\times (2l_a+1)(2l+1) \sum_L\frac{i_L(2\mu r_1 r_2)}{\exp(2\mu r_1 r_2)}
\begin{Bmatrix} L & l & l_a \\ l & j & \frac{1}{2} \\
l_a & \frac{1}{2} & j_a \end{Bmatrix}
\Big(C_{l_a 0 l 0}^{L0}\Big)^2 \qquad \textbf{[--]}
\end{align}

The fourth term is similar
\begin{align}
\sum_m &\big[\psi_a^\dagger(r_1\sigma_1)\psi_{\bar{d}}(r_1\sigma_2)\big]V_{12}
\big[\psi_{\bar{b}}^\dagger(r_2\sigma_2)\psi_c(r_2\sigma_1)\big]
\kappa_{c\bar{d}}^{(j,l)} = \nonumber\\
&\quad =\kappa_{c\bar{d}}^{(j,l)} \int_0^\infty r_1^2\mathrm{d}r_1
\int_0^\infty r_2^2\mathrm{d}r_2 R_a(r_1) R_d(r_1) \mathrm{e}^{-\mu(r_1-r_2)^2}
R_b(r_2) R_c(r_2) \sum_{LM}\frac{i_L(2\mu r_1 r_2)}{\exp(2\mu r_1 r_2)}
\frac{2l+1}{2L+1} \nonumber\\
&\qquad\times\sqrt{(2l_a+1)(2l_b+1)}\,
C_{l_a 0 l 0}^{L0} C_{l_b 0 l 0}^{L0} \sum_{m\sigma_1\sigma_2}
(-1)^{m_a-m+l+l_b+j+j_b+m+m_b} \nonumber\\
&\qquad\times C_{l_a,m_a-\sigma_1,\frac{1}{2},\sigma_1}^{j_a,m_a}
C_{l,-m-\sigma_2,\frac{1}{2},\sigma_2}^{j,-m}
C_{l_a,-m_a+\sigma_1,l,-m-\sigma_2}^{L,M}
C_{l_b,-m_b-\sigma_2,\frac{1}{2},\sigma_2}^{j_b,-m_b}
C_{l,m-\sigma_1,\frac{1}{2},\sigma_1}^{j,m}
C_{l_b,m_b+\sigma_2,l,m-\sigma_1}^{L,-M} \nonumber\\
&\quad =(2j+1)\kappa_{c\bar{d}}^{(j,l)} \int_0^\infty r_1^2\mathrm{d}r_1
\int_0^\infty r_2^2\mathrm{d}r_2 R_a(r_1) R_d(r_1) \mathrm{e}^{-\mu(r_1-r_2)^2}
R_b(r_2) R_c(r_2) \nonumber\\
&\qquad\times (2l_a+1)(2l+1) \sum_L\frac{i_L(2\mu r_1 r_2)}{\exp(2\mu r_1 r_2)}
\begin{Bmatrix} L & l & l_a \\ l & j & \frac{1}{2} \\
l_a & \frac{1}{2} & j_a \end{Bmatrix}
\Big(C_{l_a 0 l 0}^{L0}\Big)^2 \qquad \textbf{[+]}
\end{align}

\section{Some considerations of HFB equations}
During the solution of Hartree-Fock-Bogoliubov task, the particle creation and annihilation operators in spherical harmonic oscillator basis $\hat{a}_a^+,\,\hat{a}_a$ are transformed into quasiparticle creation and annihilation operators $\hat{\beta}_k^+,\,\hat{\beta}_k$, so that
\begin{equation}
\hat{\beta}_k|\mathrm{HFB}\rangle = 0, \qquad
\hat{\beta}_k^+|\mathrm{HFB}\rangle = |k\rangle, \qquad
\langle k|\hat{H}|k \rangle -
\langle\mathrm{HFB}|\hat{H}|\mathrm{HFB}\rangle = E_k
\end{equation}
The transformation between particles and quasiparticles is unitary
\begin{equation}
\begin{array}{ll}
\displaystyle\hat{\beta}_k^+ = \sum_a U_{ak}\hat{a}_a^+ + V_{ak}\hat{a}_a \qquad &
\displaystyle\hat{a}_a^+ = \sum_k U_{ak}^*\hat{\beta}_k^+ + V_{ak}\hat{\beta}_k \\
\displaystyle\hat{\beta}_k = \sum_a V_{ak}^*\hat{a}_a^+ + U_{ak}^*\hat{a}_a &
\displaystyle\hat{a}_a = \sum_k V_{ak}^*\hat{\beta}_k^+ + U_{ak}\hat{\beta}_k
\end{array}
\end{equation}
Density matrices are then
\begin{equation}
\rho_{ab} = \langle\mathrm{HFB}|\hat{a}_b^+\hat{a}_a|\mathrm{HFB}\rangle
= \sum_k V_{bk} V_{ak}^*, \qquad
\kappa_{ab} = \langle\mathrm{HFB}|\hat{a}_b\hat{a}_a|\mathrm{HFB}\rangle
= \sum_k U_{bk} V_{ak}^*
\end{equation}
The ground state energy is
\begin{equation}
E_0 = \sum_{ab} t_{ab} +
\frac{1}{2}\sum_{abcd} \bar{v}_{acbd}\rho_{ba}\rho_{dc} +
\frac{1}{4}\sum_{abcd} \bar{v}_{abcd}\kappa_{ab}^*\kappa_{cd}
\end{equation}
The HFB equations (with particle-number constrain) can be formulated as an eigenvalue problem
\begin{equation}
\label{HFB_eq}
\begin{pmatrix} h-\lambda & \Delta \\ -\Delta^* & -h^*+\lambda \end{pmatrix}
\begin{pmatrix} U_k \\ V_k \end{pmatrix} = \begin{pmatrix} U_k \\ V_k \end{pmatrix} E_k
\end{equation}

Since our system has spherical symmetry, the matrices and coefficients are real and the only non-zero terms are (I consider $m_a = m_b$ in each term)
\begin{equation}
h_{ab} = h_{\bar{a}\bar{b}},\quad \Delta_{a\bar{b}} = -\Delta_{\bar{a}b},\quad
U_{ak} = U_{\bar{a}\bar{k}},\quad V_{a\bar{k}} = -V_{\bar{a}k},\quad
\rho_{ab} = \rho_{\bar{a}\bar{b}},\quad \kappa_{a\bar{b}} = -\kappa_{\bar{a}b}
\end{equation}
Similarly as in the previous section, I will prefer terms $h_{ab},\,\Delta_{a\bar{b}},\,\rho_{ab},\,\kappa_{a\bar{b}}$ in the final expressions. So for example
\begin{equation}
\hat{\beta}_k^+ = \sum_a U_{ak}\hat{a}_a^+ - V_{a\bar{k}}\hat{a}_{\bar{a}},\quad
\rho_{ab} = \sum_k V_{a\bar{k}}V_{b\bar{k}},\quad
\kappa_{a\bar{b}} = \sum_k V_{a\bar{k}} U_{bk}
\end{equation}
The HFB equation (\ref{HFB_eq}) can be separated into blocks
\begin{equation}
\begin{pmatrix} h_{ab}-\lambda & 0 & 0 & \Delta_{a\bar{b}} \\
0 & h_{\bar{a}\bar{b}}-\lambda & \Delta_{\bar{a}b} & 0 \\
0 & -\Delta_{a\bar{b}} & -h_{ab}+\lambda & 0 \\
-\Delta_{\bar{a}b} & 0 & 0 & -h_{\bar{a}\bar{b}}+\lambda \end{pmatrix}
\begin{pmatrix} U_{bk} \\ 0 \\ 0 \\ V_{\bar{b}k} \end{pmatrix} \textrm{or}
\begin{pmatrix} 0 \\ U_{\bar{b}\bar{k}} \\ V_{b\bar{k}} \\ 0 \end{pmatrix} =
\begin{pmatrix} U_{ak} \\ 0 \\ 0 \\ V_{\bar{a}k} \end{pmatrix} \textrm{or}
\begin{pmatrix} 0 \\ U_{\bar{a}\bar{k}} \\ V_{b\bar{k}} \\ 0 \end{pmatrix}
E_{k,\bar{k}}
\end{equation}
Its solution is numerically equivalent to the solution of
\begin{equation}
\begin{pmatrix} h_{ab}-\lambda & -\Delta_{a\bar{b}} \\
-\Delta_{a\bar{b}} & -h_{ab}+\lambda \end{pmatrix}
\begin{pmatrix} U_{bk} \\ V_{b\bar{k}} \end{pmatrix} =
E_k \begin{pmatrix} U_{ak} \\ V_{a\bar{k}} \end{pmatrix}
\end{equation}

\end{document}
